% Options for packages loaded elsewhere
\PassOptionsToPackage{unicode}{hyperref}
\PassOptionsToPackage{hyphens}{url}
\documentclass[
]{article}
\usepackage{xcolor}
\usepackage[margin=1in]{geometry}
\usepackage{amsmath,amssymb}
\setcounter{secnumdepth}{-\maxdimen} % remove section numbering
\usepackage{iftex}
\ifPDFTeX
  \usepackage[T1]{fontenc}
  \usepackage[utf8]{inputenc}
  \usepackage{textcomp} % provide euro and other symbols
\else % if luatex or xetex
  \usepackage{unicode-math} % this also loads fontspec
  \defaultfontfeatures{Scale=MatchLowercase}
  \defaultfontfeatures[\rmfamily]{Ligatures=TeX,Scale=1}
\fi
\usepackage{lmodern}
\ifPDFTeX\else
  % xetex/luatex font selection
\fi
% Use upquote if available, for straight quotes in verbatim environments
\IfFileExists{upquote.sty}{\usepackage{upquote}}{}
\IfFileExists{microtype.sty}{% use microtype if available
  \usepackage[]{microtype}
  \UseMicrotypeSet[protrusion]{basicmath} % disable protrusion for tt fonts
}{}
\makeatletter
\@ifundefined{KOMAClassName}{% if non-KOMA class
  \IfFileExists{parskip.sty}{%
    \usepackage{parskip}
  }{% else
    \setlength{\parindent}{0pt}
    \setlength{\parskip}{6pt plus 2pt minus 1pt}}
}{% if KOMA class
  \KOMAoptions{parskip=half}}
\makeatother
\usepackage{graphicx}
\makeatletter
\newsavebox\pandoc@box
\newcommand*\pandocbounded[1]{% scales image to fit in text height/width
  \sbox\pandoc@box{#1}%
  \Gscale@div\@tempa{\textheight}{\dimexpr\ht\pandoc@box+\dp\pandoc@box\relax}%
  \Gscale@div\@tempb{\linewidth}{\wd\pandoc@box}%
  \ifdim\@tempb\p@<\@tempa\p@\let\@tempa\@tempb\fi% select the smaller of both
  \ifdim\@tempa\p@<\p@\scalebox{\@tempa}{\usebox\pandoc@box}%
  \else\usebox{\pandoc@box}%
  \fi%
}
% Set default figure placement to htbp
\def\fps@figure{htbp}
\makeatother
\setlength{\emergencystretch}{3em} % prevent overfull lines
\providecommand{\tightlist}{%
  \setlength{\itemsep}{0pt}\setlength{\parskip}{0pt}}
\usepackage{booktabs}
\usepackage{longtable}
\usepackage{float}
\usepackage{array}
\usepackage{setspace}
\doublespacing
\usepackage{geometry}
\geometry{margin=1in}
\usepackage{fontspec}
\setmainfont{Times New Roman}
\usepackage{fancyhdr}
\pagestyle{fancy}
\fancyhf{}
\fancyhead[R]{\thepage}
\renewcommand{\headrulewidth}{0pt}
\fancypagestyle{plain}{\fancyhf{}\fancyhead[R]{\thepage}}
\usepackage{titling}
\usepackage{booktabs}
\usepackage{longtable}
\usepackage{array}
\usepackage{multirow}
\usepackage{wrapfig}
\usepackage{float}
\usepackage{colortbl}
\usepackage{pdflscape}
\usepackage{tabu}
\usepackage{threeparttable}
\usepackage{threeparttablex}
\usepackage[normalem]{ulem}
\usepackage{makecell}
\usepackage{xcolor}
\usepackage{bookmark}
\IfFileExists{xurl.sty}{\usepackage{xurl}}{} % add URL line breaks if available
\urlstyle{same}
\hypersetup{
  pdftitle={Toward a More Useful Utility Analysis: A Literature Review and Web Application},
  pdfauthor={Christopher Castille},
  pdfkeywords={utility analysis, human resource management, decision
making, web application, science-practice gap, organizational
psychology},
  hidelinks,
  pdfcreator={LaTeX via pandoc}}

\title{Toward a More Useful Utility Analysis: A Literature Review and
Web Application}
\usepackage{etoolbox}
\makeatletter
\providecommand{\subtitle}[1]{% add subtitle to \maketitle
  \apptocmd{\@title}{\par {\large #1 \par}}{}{}
}
\makeatother
\subtitle{Bridging the Science-Practice Gap in Human Resource Decision
Making}
\author{Christopher Castille}
\date{December 2024}

\begin{document}
\maketitle
\begin{abstract}
Scholars continue to discuss the merits and drawbacks of utility
analysis (UA) as a tool for influencing management decisions. Despite
its potential for quantifying the economic value of human resource
interventions, UA has faced criticism for being overly complex and
potentially backfiring when presented to managers. This paper
synthesizes the literature on communicating validity and utility
information to managers, introduces the UA+ web application, and
demonstrates its usefulness through published case studies. The UA+ app
incorporates evidence-based features including expectancy charts,
plain-text descriptions, framing effects, and economic adjustments to
enhance managerial decision-making. We illustrate the app's capabilities
using three published examples: Latham and Whyte's (1994) staffing case,
Avolio et al.'s (2010) leadership development analysis, and Schmidt's
(2012) goal setting intervention. Our work contributes to bridging the
science-practice gap by providing a practical tool that leverages
research insights to help industrial-organizational psychologists and
business leaders communicate the value of theory-based interventions.
\end{abstract}

\section{Abstract}\label{abstract}

Scholars continue to discuss the merits and drawbacks of utility
analysis (UA) as a tool for influencing management decisions. Despite
its potential for quantifying the economic value of human resource
interventions, UA has faced criticism for being overly complex and
potentially backfiring when presented to managers. This paper
synthesizes the literature on communicating validity and utility
information to managers, introduces the UA+ web application, and
demonstrates its usefulness through published case studies. The UA+ app
incorporates evidence-based features including expectancy charts,
plain-text descriptions, framing effects, and economic adjustments to
enhance managerial decision-making. We illustrate the app's capabilities
using three published examples: Latham and Whyte's (1994) staffing case,
Avolio et al.'s (2010) leadership development analysis, and Schmidt's
(2012) goal setting intervention. Our work contributes to bridging the
science-practice gap by providing a practical tool that leverages
research insights to help industrial-organizational psychologists and
business leaders communicate the value of theory-based interventions.

\section{Author Note}\label{author-note}

Correspondence concerning this article should be addressed to
Christopher Castille, Department of Management, Nicholls State
University, Thibodaux, LA. Email:
\href{mailto:christopher.castille@nicholls.edu}{\nolinkurl{christopher.castille@nicholls.edu}}

\setcounter{page}{2}

\section{Introduction}\label{introduction}

Scholars routinely bemoan the fact that management decisions are
suboptimal (Fisher et al., 2021; Highhouse, 2008; Rynes et al., 2002).
Causes have been posited such as the poor training of managers
(Highhouse, 2008; Rynes, 2012; Rynes et al., 2018), the widening
science-practice gap (Rynes, 2012; Pfeffer \& Fong, 2002), and fetish
for theory in research that emphasizes complexity at the expense of
validity and utility in our published research (Götz \& O'Boyle, 2023).
To address these concerns, educators in human resource management and
industrial-organizational psychology have been encouraged to improve
their training of both practitioners and scholarly researchers using
methods that emphasize validity and utility of theory-based
interventions (Sturman, 2000, 2001; Russell, 2022).

Utility analysis (UA) is one method of describing such evidence in
statistical, economic, and heuristic terms (i.e., quantity, quality, and
cost; Cascio et al., 2019). Recent examples of scholars using utility
analysis to communicate the value of theory-based interventions include
Avolio et al.~(2010) who discuss the merits of a transformational
leadership intervention; Schmidt (2012) who highlights the economic
impact of goal setting; and Oprea et al.~(2019) who highlight the value
of a brief job crafting intervention. Consider a situation where several
hundred workers must be hired into a firm. UA can be used to identify
the highest quality of workers that can be reasonably delivered to the
organization for a given cost (De Corte, 2011). An example is Sturman
(2000) who used UA to demonstrate that an enhanced employee program
described by a validity coefficient of .4 and applied to over 1400
applicants to hire 470 workers hired over several years is (after
accounting for a series of adjustments) expected to yield a median
return \textasciitilde\$1.7M experienced over the lifetime use of the
program. Although large, this value represents only a few thousand
dollars per hire per year. Such economic value represents potential
revenue enhancements or cost savings attributed to, in this case, hiring
higher-performing workers (Schmidt, 2012).

Although UA is widely viewed as a complicated tool to both teach and
use, complicated tools in finance (e.g., the Black-Scholes equation)
also enjoy wide-spread use (Sturman, 2000). Russell (2022) notes that he
often builds web apps for IO psychology teams to use UA to aid
management decision making. What seems to be needed are tools that help
decision-makers (both IOs and non-IOs alike) make use of UA insights to
optimize investments more effectively in people.

With our manuscript, we make two contributions to the literature. First,
we synthesize the literature on communicating validity and utility
information to managers. Second, we build off both Cascio et al.'s
(2019) (who designed a web app made available in their \emph{Investing
in People} textbook) and Russell (2022) (who designs web apps for UA in
practice) by introducing the UA+ web app, a tool designed to leverage
insights from literature to help IOs and business leaders communicate
the validity and utility of theory-derived interventions. In the
sections to follow, we summarize the literature informing the design of
our app, explain the capabilities that the UA+ application offers over
what is currently available, offer a tutorial grounded in the
literature, and conclude by discussing future directions for functions
built into the app. We hope the IO psychology community finds our app
useful while identifying ways to enhance it and that scholars use it to
teach students in applying UA to managerial decisions.

\section{Method}\label{method}

\subsection{A Review of the Utility Analysis (UA)
Literature}\label{a-review-of-the-utility-analysis-ua-literature}

UA research goes back to the 1970s (see Boudreau, 1988). The basic
utility model calculates utility as the dollar value of performance
differences resulting from an intervention (e.g.~higher performance from
better selection or training) minus the costs of the intervention. The
general formula is below:

\[\Delta U = T \times N \times (Z_{barx} \times r \times SD_y) - C\]

Where:

\begin{itemize}
\tightlist
\item
  \(\Delta U\) = Utility change from selection device
\item
  \(N\) = Number of people hired
\item
  \(T\) = Average tenure of those hired
\item
  \(Z_{barx}\) = Average Z-score of predictor for hired employees
\item
  \(r\) = Correlation between predictor and criterion (e.g.,
  performance)
\item
  \(SD_y\) = Dollar value of a SD change in criterion
\item
  \(C\) = Cost of the intervention
\end{itemize}

Unfortunately, this basic equation yields grossly inflated estimates of
economic value (e.g., over 14,000\% return on investment; see Sturman,
2000).

Unsurprisingly, a `futility of utility' research vein emerged positing
that presenting UA information to managers backfires (Latham \& Whyte,
1994; Whyte \& Latham, 1997). In a series of studies, Latham and
colleagues examined whether presenting utility information (e.g., the
economic value of enhanced staffing) supplemented the presentation of
validity information (e.g., the accuracy of the pre-employment staffing
test). Latham et al.~found that while presenting validity information
was received favorably, presenting utility information alongside
validity information backfired, making managers hesitant to adopt a
valuable intervention. This narrative is discussed to this day (e.g.,
van Iddekinge et al., 2023) and for good reason; practitioners should be
rightly concerned about using methods that could backfire.

However, we should note that in explaining the `futility of utility,'
Cronshaw (1997) (who was a contributor to the `futility of utility'
research vein) theorized that the `backfire effect' is attributable to a
`persuasional hypothesis'. Specifically, management is right to be
skeptical of overly complex methods for the problem at hand. Cronshaw
went on to posit that utility information can help managers make more
optimal decisions about practices that impact the workforce when it is
presented to inform rather than to persuade (Cronshaw, 1997).

Accordingly, later replication efforts showed more promise presenting
utility with validity information. Carson et al.~(1998) found that
managers' acceptance of UA information was positively impacted by
utility information, although acceptance of the proposed intervention
was still distressingly low. Later work by Macan and Foster (2004) found
managers ranked UA information (e.g., the economic value of an
intervention) highly when making final decisions, stating the dollar
values helped in this. Later, Brooks et al.~(2014) found using
non-traditional effect size indicators such as common language effect
size (CLES) helps managers understand the validity of interventions.

We should note that scholars have provided guidance for making utility
estimates more accurate that include (i) accounting for economic factors
(i.e., interest, taxes, variable costs), (ii) (in the context of
selection) the cost of using multiple selection devices, deviating from
top-down hiring, and using a probationary period, and (iii) cohort or
temporal effects (e.g., validity over time) (see Sturman, 2000).
Although such adjustments can have a cumulative effect of shrinking
utility estimates to levels that are \textasciitilde4\% of the basic
utility analysis estimate and further complicate UA, they can help
address flaws in assumptions of the basic model and provide a more
realistic valuation of human resource programs for organizations.
Unfortunately, managers' receptivity to these methods has not been
examined empirically.

In sum, we believe that used thoughtfully, UA can serve as a useful tool
to help business leaders make more informed decisions about investing in
people. Although it has been noted that UA can be helpful where IO
psychologists in their context (Russell, 2022), we believe that as a
discipline we can always improve how we help both business leaders and
academicians consider the implications of theory-based interventions.

\section{Results}\label{results}

\subsection{Introducing the Utility Analysis Plus (UA+)
Tool}\label{introducing-the-utility-analysis-plus-ua-tool}

We introduce the Utility Analysis plus (UA+) tool. Our work builds on
Cascio et al.~(2019) by incorporating insights from validity and utility
analysis research that have either been shown to help managers make
decisions regarding how best to staff, train, and develop their talent
or are believed to support management decision-making (see Table 1).
Also is a glossary of terms for those less familiar with UA, and ways to
export the analysis in the .pdf format. Below, we offer a tutorial of
the app using three published examples: (i) the classic case of Latham
and Whyte (1994) (adapted by Carson et al., 1998), (ii) using UA to
estimate returns on leadership development investment (Avolio et al.,
2010), and (iii) using UA to communicate goal setting intervention value
(Schmidt, 2012). We do not cover all features of the app here as several
are still under development (e.g., using Monte Carlo analysis to
facilitate risk management). The app can be accessed here:
\url{https://uaplus.shinyapps.io/UAPlus/}.

\begin{table}[!h]
\centering
\caption{\label{tab:literature-table}Literature Review Summary of UA Attributes and Their Inclusion in Tools}
\centering
\begin{tabular}[t]{lllll}
\toprule
UA Attribute & Expected Impact & Investing in People Online & UA+ & Description\\
\midrule
\cellcolor{gray!10}{Validity} & \cellcolor{gray!10}{Positive but weak} & \cellcolor{gray!10}{✓} & \cellcolor{gray!10}{✓} & \cellcolor{gray!10}{Validity should help managers see that practices work as claimed}\\
Utility & Mixed & ✓ & ✓ & Utility should help managers see economic value\\
\cellcolor{gray!10}{SDy} & \cellcolor{gray!10}{Mixed} & \cellcolor{gray!10}{✓} & \cellcolor{gray!10}{✓} & \cellcolor{gray!10}{SDy helps managers appreciate economic value of performance}\\
Break-even levels & Unclear & ✓ & ✓ & Break-even values help managers see minimum economic value needed\\
\cellcolor{gray!10}{Expectancy Charts} & \cellcolor{gray!10}{Mixed} & \cellcolor{gray!10}{} & \cellcolor{gray!10}{✓} & \cellcolor{gray!10}{Expectancy charts help managers see validity-probability relationships}\\
\addlinespace
Framing Effects & Unclear &  & ✓ & Framing utility as opportunity costs vs. monetary gains\\
\cellcolor{gray!10}{Plain-Text Descriptions} & \cellcolor{gray!10}{Positive} & \cellcolor{gray!10}{} & \cellcolor{gray!10}{✓} & \cellcolor{gray!10}{Plain-text descriptions improve comprehension}\\
\bottomrule
\end{tabular}
\end{table}

\subsection{Case Examples}\label{case-examples}

\subsubsection{Improving Staffing: The Latham \& Whyte (1994)
Example}\label{improving-staffing-the-latham-whyte-1994-example}

\subsubsection{Improving Training or Employee
Development}\label{improving-training-or-employee-development}

\paragraph{Case 1: The Avolio et al.~(2010)
Example}\label{case-1-the-avolio-et-al.-2010-example}

\paragraph{Case 2: The Schmidt (2012)
Example}\label{case-2-the-schmidt-2012-example}

The code for this app is available via Github:
\url{https://github.com/utilityanalysis/webApp}. Interested readers are
welcome to issue requests for the app or improve the app as they like
for their purposes.

\section{Discussion}\label{discussion}

We have given a brief overview of the UA+ web app. In closing, we wish
to share our plans for (i) adding more features to the app and (ii)
engaging in a set of literal replications to identify those features
that meaningfully enhance the usefulness of this app for our community.

\subsection{Planned Features}\label{planned-features}

Features we plan to include follow. First, we wish to add graphical
descriptions of UA (i.e.~slides) that allow a causal chain to be
described (e.g., \% increase in training → \% increase in performance →
\% increase in economic value), which Winkler et al.~(2010) suggest help
managers see how an intervention impacts the workforce. Second,
concerning the accuracy of UA, Sturman (2000) notes that a Monte Carlo
analysis allows a comprehensive set of adjustments to be used to
accurately appraise the value of HR interventions. We hope to update the
UA+ app with a Monte Carlo analysis feature that is relatively easy to
use and reproduces (as closely as possible) the findings from Sturman
(2000) (e.g., adjustments reduce basic utility estimates to
\textasciitilde4\% of their initial value). Third, we wish to augment
the selection utility tools in the app with a pareto-optimization
feature that allows users to identify the most economic value that can
be attained while diversifying the workforce (see De Corte et al., 2011;
Rupp et al., 2020). Fourth, we wish to highlight the neglected role of
compensation in utility research (e.g., its impact on workforce value,
see Sturman, 2001). Our long-term aim is to have a tool that allows HR
professionals to identify optimal bundles of interventions (e.g.,
combinations of staffing, training, and compensation enhancements).

\subsection{Future Research
Directions}\label{future-research-directions}

We wish to close by noting that while our app's design is grounded in
the literature, more information or features in the app may not always
be better (Connelly et al., 2023); sometimes, less is more. Indeed, as
described in Table 1, there are many claims concerning how to present
evidence to managers, some of which have been supported. Therefore, our
first step is to replicate those effects that have been tested in the
literature via a set of literal replication studies (i.e., a replication
aimed at replicating, as close as possible, the initial conditions of
these studies). In this first wave of confirmatory testing, we will
identify and retain information features that are useful for our
community. As part of this, we will include conditions that test claims
that have gone relatively untested (e.g., framing effects applied to
validity). Our aim is to provide clear evidence concerning what does and
does not help users make decisions regarding investing in their people
and then build this information in the UA+ app. We have already begun
pre-registering our methods and hypotheses on the Open Science Framework
and look forward to incorporating feedback from our reviewers.

\section{Conclusion}\label{conclusion}

This paper has presented the UA+ web application as a practical tool for
bridging the science-practice gap in utility analysis. By synthesizing
the literature on communicating validity and utility information to
managers, we have identified key features that enhance managerial
decision-making, including expectancy charts, plain-text descriptions,
framing effects, and economic adjustments. The UA+ app incorporates
these evidence-based features to help industrial-organizational
psychologists and business leaders communicate the value of theory-based
interventions more effectively.

Our demonstration of the app using three published case studies---Latham
and Whyte's (1994) staffing analysis, Avolio et al.'s (2010) leadership
development study, and Schmidt's (2012) goal setting
intervention---illustrates the app's versatility across different HR
interventions. The app successfully reproduces key findings from these
studies while providing additional features that enhance comprehension
and decision-making.

Future development of the UA+ app will focus on incorporating Monte
Carlo analysis capabilities, pareto-optimization features, and
compensation considerations. Additionally, planned replication studies
will systematically test the effectiveness of various presentation
features to ensure that the app provides optimal support for managerial
decision-making.

The UA+ app represents a step toward addressing the persistent challenge
of translating research insights into practical tools that can enhance
organizational decision-making. By providing a user-friendly interface
that incorporates evidence-based features, we hope to contribute to the
broader goal of improving how organizations invest in their people and
make evidence-based decisions about human resource interventions.

\section{References}\label{references}

\end{document}
